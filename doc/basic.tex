%basic.tex
%This is the latex documentation for my BASIC language.
\documentclass{report}
\usepackage{listings}

\begin{document}

%The title
\title{BASIC}
\author{patrickf200}
\maketitle

\tableofcontents

%================================================
%Introduction
\chapter{Welcome to BASIC!}

Welcome to the documentation for this BASIC interpreter. This documentation explains how the interpreter and the language works. While we try to add complete documentation here, you should refer to some of the examples in the 'examples' directory. Additionally, I recommend that you open a BASIC shell and give some of these commands a try as you read.
\newline

\noindent
I hope you have as much fun using this interpreter as I did making it. Enjoy!
\newline

\section{Language changes:}

I will generally try to avoid drastic language changes, as that creates a mess among existing programs. However, it may occassionally be necessary in order to make things easier. As far as backwards compatibility goes, while it depends on the feature, I may or may not retain the old way of doing something. Backwards compatibility often leads to software bloat, and can discourage changing to the new way. Ultimately, it will depend on the change. If anything, I hope the changes will be in the way of new features.

%================================================
%Using the interpreter
\chapter{Using the Interpreter}
The interpreter can be used in two ways: Shell mode and File mode. All commands are valid both in Shell mode and File mode, although some are meant more for shell mode.

\section{Shell Mode}
To use shell mode, simply run the program (the binary is called 'basic'). When you do so, you should be greeted with the a prompt that looks something like this:

\begin{lstlisting}
BASIC>
\end{lstlisting}

\noindent This is where you type your commands. All commands begin with a capital letter, and are case sensitive. If you wish to create a function (let's say its called 'Hello'), you will notice that the prompt will look something like this:

\begin{lstlisting}
BASIC> [Hello]
\end{lstlisting}

\noindent To change the prompt title, you can use the 'Program' command, like this:

\begin{lstlisting}
BASIC> Program Hello
Hello>
\end{lstlisting}

\noindent To exit the interpreter, use the 'Exit' command.

\section{File Mode}
File mode is very easy. To do so, simply run the basic binary, followed by the path to your file name (if the file is in the current working directory, just the file name will do). Basic currently takes no command line parameters. If you want author and license information, go to shell mode and use the 'Author' command. To see version information, use the 'Version' command. Note that you are not limited to just one file. In file mode, you can pass as many programs to the binary as you wish; they will be executed one at a time.

%================================================
%Program structure/functions
\chapter{Program structure/Functions}

\section{Program Structure}

Like many interpreted languages, BASIC does not require everything to be in a function. In fact, at least part of every BASIC program must NOT be part of a function. When BASIC sees a function, it just records the function and makes a reference to it. It will never be run until you use the 'Call' command. Like many languages, function order does matter. You must define your functions before you call them. See the next section for more information.

\section{Functions}

Functions can be used in both shell and file mode. However, they will only be recorded unless you call them. To define a function, first use the 'Function' keyword, followed by its name. Then, use the keyword 'Begin'. If you are in shell mode, you will notice the function name in brackets next to the prompt. To indicate the end of a function, use the 'End' keyword. When you are ready to use it, use the 'Call' keyword. Below is an example. You may use it in either shell or file mode.

\begin{lstlisting}
Function SayHello
Begin
	PrintLine "Hello!"
End

Call SayHello
\end{lstlisting}

As of version 1.3, functions can take parameters like other languages. To define parameters, simply end your function name with an opening bracket, followed by your parameters separated by commas, and ending with a closing bracket. Passing parameters is very similar. When you use the 'Call' command, end the function name with an opening bracket, followed by your input parameters. Parameters can be either variables are hard-coded. If your function takes no parameters, then use no brackets. Below is an example:

\begin{lstlisting}
Function PrintMsg[msg1,msg2]
Begin
	PrintLine "Msg1:"
	PrintLine msg1
	
	PrintLine "Msg2:"
	PrintLine msg2
End

Var message
Define message as "Hello!"

Call PrintMsg[message,"How are you?"]
\end{lstlisting}

%================================================
%Variables/local memory
\chapter{Variables/Local Memory}

\section{Introduction to Variables and Local Memory}

Depending on what kind of a programming background you come from, you may or may not find variables and the concept of local memory to be a little wierd. Hopefully not. Like most programming languages, variables exist on the local and global scope (prior to BASIC version 1.2, they only existed globally). Global variables can be accessed anywhere in the program, including functions, after they have been declared. Local variables are declared in a function, and destroyed when the function returns.

Local memory is nothing special. It is simply a static variable in the interpreter that can be controlled by the programmer. Many commands return their value to local memory. Local memory is useful when you want to temporarily set or pass a value without creating an entire variable.

\section{Creating and defining variables}

Defining variables is easy, but I will admit that it is a little more cumbersome than in other languages. This has been changed slightly to make it easier, though.

Creating variables: To create variables, use the 'Var' keyword, followed by a name. You do not define the variable here; all you are doing is creating it. If a variable by that name already exists, it will not be re-created; nothing will happen.

Defining variables: Defining variables is used by the 'Define' command. The syntax is: Define (var name) as (value). String values must be enclosed in quotation marks. To set the result of a command, use the syntax as above, only enclose your command in parentheses. Variables can also be defined by using local memory. See below in the section on local memory.

\section{Using local memory}

Local memory, conceptually speaking, is not that different from variables. The main difference is that only one form of it exists, and it must be copied to a variable to be used. However, local memory is not read-only to the user; you are free to set it as well.

To set local memory, use the 'Memset' command, like this: Memset (value)

To copy local memory to a variable, use the 'Set' command, like this: Set (variable name)

To clear local memory, you can use one of two commands. First, you can use 'Memset', like this: Memset "". Or, you can use the 'Destroy' command, which we will cover next.

\section{Kill and Destroy}

The 'Kill' command is used to completely erase and remove variables from the global variable array. Simply use it like this: Kill (var name)

The 'Destroy' command is used to wipe memory. It takes one of three parameters: vars, lists, mem, all. The 'vars' parameter clears all variales. The 'lists' parameter clears all lists (including the command line arguments list, 'args'). The 'mem' parameter will reset local memory. The 'all' parameter will combine the other two.

%================================================
%Lists
\chapter{Lists}

\section{Introduction}

BASIC has support for lists, commonly called arrays in other languages. While similar to variables, Lists currently exist only in the global scope. Unlike other commands, there is only one List command, with multiple options.

\section{Syntax}

\begin{itemize}
\item{Creating lists:}\newline
List (list name)

\item{Adding items to lists:}\newline
List Add (var,value) to (list name)

\item{Getting list length:}\newline
List Length of (list name)\newline
Note: The value is saved to memory.

\item{Getting an item in a list:}\newline
List Show (index) in (list name)

\item{Removing an item from a list:}\newline
List Remove (index) in (list name)

\item{Clearing lists:}\newline
List Clear (list name)

\item{Destroying list variables:}\newline
List Kill (list name)

\end{itemize}

%================================================
%Conditionals
\chapter{Conditionals}

\section{Overview}

Like all other programming languages, BASIC has support for conditionals. Conditionals are a little different from other languages, and may take a little getting-used to. Below is an example of a conditional (it asks the user for two numbers and compares them):

\begin{lstlisting}
Var no1
Var no2

Define no1 as 0
Define no2 as 0

Input no1
Input no2

If no1 Greater no2
	PrintLine "Greater!"
Else no1 Less no2
	PrintLine "Less!"
Else no2 Equals no2
	PrintLine "Equal!"
Else
	PrintLine "What did you enter?"
Stop
\end{lstlisting}

As you can see, the general idea is the same as in other languages; its just the implementation that is slightly different. You can compare numbers and strings, both hard-coded or in variables. The first statement always begins with If. All other statements begin with Else. If you want an else with no condition, simply use 'Else'. Else, with a condition, is the equivalent of 'else if' in other languages. Finally, end the entire statement with 'Stop'.

\section{Operators}

Unlike most other programming languages, BASIC does not use symbolic operators; we use words instead. The three operators are 'Greater', 'Less', and 'Equals'. What they do should be self-explanatory.

\section{Nesting Loops}

This is important. You CANNOT nest conditionals in BASIC. If you try to do so, you will create a conflict, and the program will get stuck (it shouldn't freeze your computer, though; Ctrl-C should kill it). If you must nest conditionals, you can enclose the next conditional statement in a function, and call that function.

%================================================
%Loops
\chapter{Loops}

\section{Introduction}

There are two different types of loops in BASIC: the simple loop and the for loop.

\section{The simple loop}

The simple loop is called by the 'Loop' command. The reason why it is not called the 'stupid loop' is because it does give you a way out. The simple loop simply runs a set of commands a certain number of times. This is the syntax: Loop (no or variable)

A simple loop is only useful if you only need to run something a certain number of times AND you do not need access to the current index. To use it, simply use the loop command, followed by the body, and ending with the 'Stop' command.

\section{The for loop}

The for loop is similar in a lot of ways to the simple loop. The main difference is that it gives you access to the current index. This is the syntax: For (var) to (var or number)
Below is an example:

\begin{lstlisting}
Var i
Define i as 0

For i to 11
PrintLine i
Stop
\end{lstlisting}

\section{Loop limitations}

Both loops have a few limitations. First, we currently have no way to break out early from a loop. Once you start a loop, you are committed, so make sure it is not infinite loop. Secondly, you CANNOT nest loops or conditionals in a loop. To circumvent this, use functions. Finally, if you use the 'Var' command in a loop, you must use the 'Kill' command so you are not constantly redifining variables and needlessly growing the variable arrary.

%================================================
%System IO
\chapter{System IO}

\section{The Console}

There are three main console operations: Printing, clearing the screen, and coloring text. You can print using one of two commands: Print and PrintLine. Print just prints what you want, but does not end with a new line. PrintLine does. Print is ideal for user input prompts and when you wish to display variables and text on one line.

Clearing the screen is done using the Clear command. This will clear your console and reset the cursor.

The Color command is used to set the text color. Use ColorHelp to get a list of available colors. Color uses ASCII escape sequences to color text, and while it should be supported by the majority of consoles and systems, you may still run into cases where it causes wierd behavior.

\section{User Input}

You can get user input from the console using the Input command. Input takes a variable as a parameter. If you want "Press any key to continue" behavior, simply provide Input without any variables.

\section{Command Line Arguments}

If you are running in file mode, you can access command line arguments. Arguments are passed to a List called "args". You can access and use this like any other list.
WARNING: If you use the List Kill or the Destroy command, you can destroy this list. If you have any command line arguments, it is advisable to use them as early as possible.

\section{File IO}

BASIC now has simple support for file operations. At this point, it can only read and write files. Files are read line by line to a List, and writing is done by passing a List (where each element represents a single line) to be read and written. Below is the syntax:

Reading files:
File Read (file name) to (list variable)

Writing files:
File Write (list variable) to (file name)

%================================================
%All Commands
\chapter{All Commands}

\begin{itemize}
\item{Print (string or var)}\newline
Print to the console (no new line)

\item{PrintLine (string or var)}\newline
Print to the console (end in new line)

\item{Exit}\newline
Exit a program

\item{Function (name)}\newline
Declare a new function

\item{Begin}\newline
Begin the body of a function

\item{End}\newline
End the body of a function

\item{Clear}\newline
Clear the console screen

\item{Comment (content)}\newline
Indicates comments in a source file.

\item{Var (name)}\newline
Create a new variable.

\item{Define (var name) as (value)}\newline
Define a variable

\item{Add (no,var) and (no,var)}\newline
Add two numbers

\item{Sub (no,var) and (no,var)}\newline
Subtract two numbers.

\item{Mp (no,var) and (no,var)}\newline
Multiply two numbers.

\item{Div (no,var) and (no,var)}\newline
Divide two numbers.

\item{Remainder (no,var) and (no,var)}\newline
Get the remainder of two numbers when divided (equivalent to modulus in most other languages).

\item{Memset (value)}\newline
Set the value of local memory.

\item{Set (var)}\newline
Set the value of local memory to a variable.

\item{StrLen (string,var)}\newline
Get the length of a string (value saved to memory).

\item{Char (char or no) in (string,var)}\newline
Gets the location of a particular character or the character a particular location. The value is saved to memory. Note that if you wish to input a variable, make sure it is at least two characters long; if not, the command will think you are looking for the location of a character.

\item{Kill (var)}\newline
Erase a variable and its value.

\item{Destroy [vars,mem,all]}\newline
Destroy all variables, the local memory, or everything.

\item{Abs (var, no)}\newline
Get the absolute value of a number.

\item{Random (no)}\newline
Generate a random number up to a certain limit (value saved to memory).

\item{Color (color name)}\newline
Set the text color in the console (may not work on all systems or terminals).

\item{ColorHelp}\newline
Get a list of available colors (for the color command).

\item{Input (var)}\newline
Prompt the user for input.

\item{Program (name)}\newline
Set the text in the shell prompt.

\item{Author}\newline
Display author and license information.

\item{Version}\newline
Display version information.
\end{itemize}

\end{document}

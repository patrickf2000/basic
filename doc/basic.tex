%basic.tex
%This is the latex documentation for my BASIC language.
\documentclass{report}
\usepackage{listings}

\begin{document}

%The title
\title{BASIC}
\author{patrickf200}
\maketitle

\tableofcontents

%================================================
%Introduction
\chapter{Welcome to BASIC!}

Welcome to the documentation for this BASIC interpreter. This documentation explains how the interpreter and the language works. While we try to add complete documentation here, you should refer to some of the examples in the 'examples' directory. Additionally, I recommend that you open a BASIC shell and give some of these commands a try as you read.
\newline

\noindent
I hope you have as much fun using this interpreter as I did making it. Enjoy!
\newline

\section{Language changes:}

I will generally try to avoid drastic language changes, as that creates a mess among existing programs. However, it may occassionally be necessary in order to make things easier. As far as backwards compatibility goes, while it depends on the feature, I may or may not retain the old way of doing something. Backwards compatibility often leads to software bloat, and can discourage changing to the new way. Ultimately, it will depend on the change. If anything, I hope the changes will be in the way of new features.

%================================================
%Using the interpreter
\chapter{Using the Interpreter}
The interpreter can be used in two ways: Shell mode and File mode. All commands are valid both in Shell mode and File mode, although some are meant more for shell mode.

\section{Shell Mode}
To use shell mode, simply run the program (the binary is called 'basic'). When you do so, you should be greeted with the a prompt that looks something like this:

\begin{lstlisting}
BASIC>
\end{lstlisting}

\noindent This is where you type your commands. All commands begin with a capital letter, and are case sensitive. If you wish to create a function (let's say its called 'Hello'), you will notice that the prompt will look something like this:

\begin{lstlisting}
BASIC> [Hello]
\end{lstlisting}

\noindent To change the prompt title, you can use the 'Program' command, like this:

\begin{lstlisting}
BASIC> Program Hello
Hello>
\end{lstlisting}

\noindent To exit the interpreter, use the 'Exit' command.

\section{File Mode}
File mode is very easy. To do so, simply run the basic binary, followed by the path to your file name (if the file is in the current working directory, just the file name will do). Basic currently takes no command line parameters. If you want author and license information, go to shell mode and use the 'Author' command. To see version information, use the 'Version' command. Note that you are not limited to just one file. In file mode, you can pass as many programs to the binary as you wish; they will be executed one at a time.

%================================================
%Program structure/functions
\chapter{Program structure/Functions}

\section{Program Structure}

Like many interpreted languages, BASIC does not require everything to be in a function. In fact, at least part of every BASIC program must NOT be part of a function. When BASIC sees a function, it just records the function and makes a reference to it. It will never be run until you use the 'Call' command. Like many languages, function order does matter. You must define your functions before you call them. See the next section for more information.

\section{Functions}

Functions can be used in both shell and file mode. However, they will only be recorded unless you call them. To define a function, first use the 'Function' keyword, followed by its name. Then, use the keyword 'Begin'. If you are in shell mode, you will notice the function name in brackets next to the prompt. To indicate the end of a function, use the 'End' keyword. When you are ready to use it, use the 'Call' keyword. Below is an example. You may use it in either shell or file mode.

\begin{lstlisting}
Function SayHello
Begin
	PrintLine "Hello!"
End

Call SayHello
\end{lstlisting}

As of version 1.3, functions can take parameters like other languages. To define parameters, simply end your function name with an opening bracket, followed by your parameters separated by commas, and ending with a closing bracket. Passing parameters is very similar. When you use the 'Call' command, end the function name with an opening bracket, followed by your input parameters. Parameters can be either variables are hard-coded. If your function takes no parameters, then use no brackets. Below is an example:

\begin{lstlisting}
Function PrintMsg[msg1,msg2]
Begin
	PrintLine "Msg1:"
	PrintLine msg1
	
	PrintLine "Msg2:"
	PrintLine msg2
End

Var message
Define message as "Hello!"

Call PrintMsg[message,"How are you?"]
\end{lstlisting}

%================================================
%Variables/local memory
\chapter{Variables/Local Memory}

\section{Introduction to Variables and Local Memory}

Depending on what kind of a programming background you come from, you may or may not find variables and the concept of local memory to be a little wierd. Hopefully not. Like most programming languages, variables exist on the local and global scope (prior to BASIC version 1.2, they only existed globally). Global variables can be accessed anywhere in the program, including functions, after they have been declared. Local variables are declared in a function, and destroyed when the function returns.

Local memory is nothing special. It is simply a static variable in the interpreter that can be controlled by the programmer. Many commands return their value to local memory. Local memory is useful when you want to temporarily set or pass a value without creating an entire variable.

\section{Creating and defining variables}

Defining variables is easy, but I will admit that it is a little more cumbersome than in other languages. This has been changed slightly to make it easier, though.

Creating variables: To create variables, use the 'Var' keyword, followed by a name. You do not define the variable here; all you are doing is creating it. If a variable by that name already exists, it will not be re-created; nothing will happen.

Defining variables: Defining variables is used by the 'Define' command. The syntax is: Define (var name) as (value). String values must be enclosed in quotation marks. To set the result of a command, use the syntax as above, only enclose your command in parentheses. Variables can also be defined by using local memory. See below in the section on local memory.

\section{Using local memory}

Local memory, conceptually speaking, is not that different from variables. The main difference is that only one form of it exists, and it must be copied to a variable to be used. However, local memory is not read-only to the user; you are free to set it as well.

To set local memory, use the 'Memset' command, like this: Memset (value)

To copy local memory to a variable, use the 'Set' command, like this: Set (variable name)

To clear local memory, you can use one of two commands. First, you can use 'Memset', like this: Memset "". Or, you can use the 'Destroy' command, which we will cover next.

\section{Kill and Destroy}

The 'Kill' command is used to completely erase and remove variables from the global variable array. Simply use it like this: Kill (var name)

The 'Destroy' command is used to wipe memory. It takes one of three parameters: vars, lists, mem, all. The 'vars' parameter clears all variales. The 'lists' parameter clears all lists (including the command line arguments list, 'args'). The 'mem' parameter will reset local memory. The 'all' parameter will combine the other two.

%================================================
%Lists
\chapter{Lists}

\section{Introduction}

BASIC has support for lists, commonly called arrays in other languages. While similar to variables, Lists currently exist only in the global scope. Unlike other commands, there is only one List command, with multiple options.

\section{Syntax}

\begin{itemize}
\item{Creating lists:}\newline
List (list name)

\item{Adding items to lists:}\newline
List Add (var,value) to (list name)

\item{Getting list length:}\newline
List Length of (list name)\newline
Note: The value is saved to memory.

\item{Getting an item in a list:}\newline
List Show (index) in (list name)

\item{Removing an item from a list:}\newline
List Remove (index) in (list name)

\item{Clearing lists:}\newline
List Clear (list name)

\item{Destroying list variables:}\newline
List Kill (list name)

\end{itemize}

\end{document}
